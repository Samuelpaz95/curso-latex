\documentclass{article}

\usepackage[utf8]{inputenc} % Indica cuál es la codificación de este archivo
\usepackage[spanish]{babel} % Indica el idioma en que está escrito el documento
\usepackage{graphicx}

\begin{document}

% Ingreso contenido de la carátula estándar de Latex
\title{Un documento de ejemplo} % Título del documento
\date{\today} % Fecha del documento. El comando \today inserta la fecha actual
\author{Ing. Huergo\thanks{FIUBA}} % Autor del documento

\maketitle % Genera la carátula

\newpage % Inicia una nueva página

\tableofcontents % Genera la tabla de contenido del documento

\newpage % Inicia una nueva página

\section*{Introducción}
\addcontentsline{toc}{subsection}{Introducción}

\section{Lorem ipsum dolor sit amet, consectetur adipiscing elit. Suspendisse ac.}

\subsection[Fusce in]{Lorem ipsum dolor sit amet, consectetur adipiscing elit. Fusce in.}

¡Hola mundo!


\end{document}
