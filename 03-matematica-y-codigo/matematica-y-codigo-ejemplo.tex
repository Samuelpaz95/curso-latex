\documentclass{article}

\usepackage[utf8]{inputenc} % Indica cuál es la codificación de este archivo
\usepackage[spanish]{babel} % Indica el idioma en que está escrito el documento
\usepackage[svgnames]{xcolor}
\usepackage{amsmath} % Provee mejoras a los entornos matemáticos y fórmulas
\usepackage{mathtools} % Correcciones a amsmath y funcionalidades extras

\begin{document}

\section{Fórmulas matemáticas}

\subsection{Entornos matemáticos}

\subsubsection{Entorno en línea}

La funcion $f(x)$ es $f(x)=ax+b$.

\subsubsection{Entorno ecuación (sin numerar)}
La ecuación

\[
  f(x)=ax+b
\]

es una función $f(x)$.

\[
  10 \text{ manzanas} + 10 \text{ manzanas} = 20 \text{ manzanas}
\]

\subsubsection{Entorno ecuación}

\begin{equation} \label{eq:ecuacion}
  f(x) = ax + b
\end{equation}

La ecuación \eqref{eq:ecuacion} es una función $f(x)$.

\subsection{Símbolos y funciones}

\[
  \cos (2\theta) = \cos^2 \theta - \sin^2 \theta
\]

\subsection{Potencias, índices, raíces y fracciones}

\subsubsection{Potencias e índices}

\begin{equation*}
  k_{n+1} = n^2 + k_n^2 - k_{n-1}
\end{equation*}

\subsubsection{Raíces}

\begin{equation*}
  \sqrt{4} = 2
\end{equation*}

\begin{equation*}
  \sqrt[3]{8} = 2
\end{equation*}

\subsubsection{Fracciones}

\begin{equation*}
  \frac{\frac{1}{x}+\frac{1}{y}}{y-z}
\end{equation*}

\subsection{Matrices}

\begin{equation*}
  \begin{matrix}
    a & b & c \\
    d & e & f \\
    g & h & i
  \end{matrix}
\end{equation*}

Matriz encerrada entre paréntesis

\begin{equation*}
  \begin{bmatrix*}[r]
    -1 & 3  \\
    2  & -4
  \end{bmatrix*}
\end{equation*}

\begin{equation*}
  A_{m,n} = 
    \begin{pmatrix}
      a_{1,1} & a_{1,2} & \cdots & a_{1,n} \\
      a_{2,1} & a_{2,2} & \cdots & a_{2,n} \\
      \vdots  & \vdots  & \ddots & \vdots  \\
      a_{m,1} & a_{m,2} & \cdots & a_{m,n}
    \end{pmatrix}
\end{equation*}

\section{Colores}

\subsection{Colores disponibles por defecto de \texttt{xcolor}}

{\color{red} Rojo}

{\color{green} Verde}

{\color{blue} Azul}

{\color{cyan} Cyan}

{\color{magenta} Magenta}

{\color{yellow} Amarillo}

{\color{black} Negro}

{\color{darkgray} Gris oscuro}

{\color{gray} Gris}

{\color{lightgray} Gris claro}

{\color{white} Blanco} (Blanco)

{\color{brown} Marron}

{\color{lime} Lima}

{\color{olive} Oliva}

{\color{orange} Naranja}

{\color{pink} Rosa}

{\color{purple} Púrpura}

{\color{teal} Verde azulado}

{\color{violet} Violeta}

\subsection{Algunos colores adicionales incluidos por la opción \texttt{svgnames}}

{\color{Aquamarine} Agua marina}

{\color{Chocolate} Chocolate}

{\color{Crimson} Crimson}

{\color{GreenYellow} Verde amarillo}

{\color{DarkBlue} Azul oscuro}

{\color{LightBlue} Azul claro}

{\color{DarkCyan} Cyan oscuro}

{\color{LightCyan} Cyan claro}

{\color{DarkGreen} Verde oscuro}

{\color{LightGreen} Verde claro}

{\color{DarkMagenta} Magenta oscuro}

{\color{DarkOrange} Naranja oscuro}

{\color{DarkRed} Rojo oscuro}

{\color{Salmon} Salmón}

{\color{SkyBlue} Azul cielo}

{\color{Tomato} Tomate}

{\color{YellowGreen} Amarillo verde}

\subsection{Usando los colores}
\subsubsection{Formateo de texto}

{\color{red} Grupo de texto con color rojo}

\textcolor{blue}{Comando con color azul para texto}

\colorbox{green}{Texto con fondo verde}

\colorbox{black}{\color{white} Texto blanco con fondo negro}

\fcolorbox{red}{white}{Texto con borde de color rojo}

\fcolorbox{blue}{green}{Texto con borde de color azul y fondo verde}


\subsubsection{Fondo de página}

\pagecolor{lightgray}

\end{document}